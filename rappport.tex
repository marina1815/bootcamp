% !TEX program = pdflatex
\documentclass[11pt,a4paper]{article}

\usepackage[margin=2.2cm]{geometry}
\usepackage[T1]{fontenc}
\usepackage[utf8]{inputenc}
\usepackage{lmodern}
\usepackage{microtype}
\usepackage{hyperref}
\usepackage[demo]{graphicx}
\usepackage{xcolor}
\usepackage{enumitem}
\usepackage{booktabs}
\usepackage{float}
\usepackage{fancyhdr}
\usepackage{titlesec}
\usepackage{listings}

\hypersetup{
  colorlinks=true,
  linkcolor=black,
  urlcolor=blue,
  citecolor=black
}

\pagestyle{fancy}
\fancyhf{}
\lhead{Bootcamp Intensif Full Stack TypeScript}
\rhead{Final Project Documentation}
\cfoot{\thepage}

\titleformat{\section}{\Large\bfseries}{\thesection}{0.7em}{}
\titleformat{\subsection}{\large\bfseries}{\thesubsection}{0.7em}{}

\definecolor{codebg}{RGB}{245,245,245}
\definecolor{codeframe}{RGB}{220,220,220}

\lstset{
  basicstyle=\ttfamily\small,
  backgroundcolor=\color{codebg},
  frame=single,
  rulecolor=\color{codeframe},
  breaklines=true,
  columns=fullflexible,
  upquote=true,
  showstringspaces=false
}

\begin{document}

\begin{titlepage}
  \centering
  \vspace*{2cm}
  {\LARGE \textbf{Final Project Documentation}\par}
  \vspace{0.6cm}
  {\Large Bootcamp Intensif Full Stack TypeScript (5 jours)\par}
  \vspace{1.2cm}

  \begin{tabular}{@{}ll@{}}
    \textbf{Participant:} & Hanaa Amira \\
    \textbf{Project:} & Doctor--Patient Appointment Platform \\
    \textbf{Stack:} & TypeScript, Node.js, PostgreSQL, Frontend Web \\
    \textbf{Date:} & \today \\
  \end{tabular}

  \vspace{1.5cm}
  \textbf{Repository (GitHub):} \\
  \href{https://github.com/marina1815/bootcamp}{https://github.com/marina1815/bootcamp}

  \vfill
  {\small This document describes the architecture, implementation, security controls, and screenshots of the delivered project.}
\end{titlepage}

\tableofcontents
\newpage

\section{Project Summary}

This project is a mini SaaS web platform that enables:
\begin{itemize}[leftmargin=1.2em]
  \item \textbf{Patients} to book appointments with doctors.
  \item \textbf{Doctors} to view their clients and manage appointments.
\end{itemize}

The implementation follows a production mindset: clear architecture, strict TypeScript typing, security controls, validation/sanitization, logging, and PostgreSQL persistence.

\section{Objectives}

The main objectives were:
\begin{itemize}[leftmargin=1.2em]
  \item Build a full stack TypeScript application (Front + Back).
  \item Implement a REST API with clean layering (controllers, services, middlewares).
  \item Enforce security best practices (JWT, RBAC, validation, rate limiting, audit logging).
  \item Provide documentation and screenshots demonstrating functionality and security behavior.
\end{itemize}

\section{Architecture Overview}

\subsection{High-level Diagram (Textual)}
\begin{center}
\begin{tabular}{@{}l@{}}
Client (Browser) \\
$\downarrow$ \\
Frontend (Login, Dashboard, Patients, Appointments) \\
$\downarrow$ \\
Backend API (Node.js + TypeScript) \\
$\downarrow$ \\
PostgreSQL Database \\
$\downarrow$ \\
Logs / Audit Trail
\end{tabular}
\end{center}

\subsection{Main Components}
\begin{itemize}[leftmargin=1.2em]
  \item \textbf{Frontend}: login flow, dashboard, forms and lists, protected routes.
  \item \textbf{Backend API}: REST endpoints, auth, RBAC, validation, logging, error handling.
  \item \textbf{Database (PostgreSQL)}: users, patients, appointments, audit logs.
\end{itemize}

\section{Technology Stack}

\begin{tabular}{@{}ll@{}}
\toprule
\textbf{Layer} & \textbf{Technology} \\
\midrule
Frontend & HTML/CSS/JS (or React/Next if applicable) \\
Backend & Node.js, TypeScript, Express \\
Database & PostgreSQL \\
Security & JWT, RBAC, bcrypt, rate limiting, validation/sanitization \\
Tooling & Bun/Node, npm, Git/GitHub \\
\bottomrule
\end{tabular}

\section{Backend Implementation}

\subsection{Folder Structure (Example)}
\begin{lstlisting}
src/
  controllers/
  services/
  routes/
  middleware/
  dto/
  models/
  db/
  utils/
\end{lstlisting}

\subsection{Data Models}
Core entities:
\begin{itemize}[leftmargin=1.2em]
  \item \textbf{User}: id, email, passwordHash, role (ADMIN / DOCTOR / ASSISTANT), createdAt.
  \item \textbf{Patient}: id, firstname, lastname, phone, createdAt.
  \item \textbf{Appointment}: id, patientId, doctorId, dateTime, status, reason.
  \item \textbf{AuditLog}: timestamp, event/action, result, actor, metadata (IP, endpoint).
\end{itemize}

\subsection{Controllers, Services, Routes}
\begin{itemize}[leftmargin=1.2em]
  \item \textbf{Controllers}: handle HTTP request/response and call services.
  \item \textbf{Services}: business logic (create appointment, list patients, etc.).
  \item \textbf{Routes}: define endpoints and apply middleware chain.
\end{itemize}

\section{Security Controls}

\subsection{Authentication (JWT)}
\begin{itemize}[leftmargin=1.2em]
  \item Login verifies credentials (password hashed with \textbf{bcrypt}).
  \item A JWT is issued with user identifier and role.
  \item Protected endpoints require a valid token.
\end{itemize}

\subsection{Authorization (RBAC)}
Role-based access control rules:
\begin{itemize}[leftmargin=1.2em]
  \item Patients management actions are restricted based on role.
  \item Audit logs access is restricted to \textbf{ADMIN}.
  \item Frontend hides restricted UI items and backend enforces restrictions (server-side).
\end{itemize}

\subsection{Validation \& Sanitization (DTOs)}
\begin{itemize}[leftmargin=1.2em]
  \item Requests are validated using strict DTO schemas.
  \item Inputs are sanitized to reduce injection/XSS risk.
  \item Invalid input returns consistent error messages.
\end{itemize}

\subsection{Rate Limiting}
\begin{itemize}[leftmargin=1.2em]
  \item Login endpoint rate limiting mitigates brute-force attempts.
  \item Example behavior: after several attempts from the same IP, requests are temporarily blocked.
\end{itemize}

\subsection{Audit Logging}
\begin{itemize}[leftmargin=1.2em]
  \item Security-relevant actions are recorded (login success/failure, access denied, etc.).
  \item Logs contain timestamps and context (actor, endpoint, IP) for traceability.
\end{itemize}

\section{Frontend Implementation}

\subsection{Pages}
\begin{itemize}[leftmargin=1.2em]
  \item \textbf{Login}: user authentication and session initiation.
  \item \textbf{Dashboard}: overview and quick actions.
  \item \textbf{Patients}: list, create, update (depending on role).
  \item \textbf{Appointments}: booking (patients) and consultation view (doctors).
  \item \textbf{Audit Logs}: visible only for ADMIN.
\end{itemize}

\subsection{Protected Routes}
Frontend routes are protected to prevent unauthorized access. Sensitive pages are hidden when role does not allow access. Backend authorization remains the source of truth.

\section{API Endpoints (Example)}

\begin{tabular}{@{}lll@{}}
\toprule
\textbf{Method} & \textbf{Endpoint} & \textbf{Description} \\
\midrule
POST & /auth/login & Authenticate user, issue JWT \\
GET & /users/me & Return authenticated user profile \\
GET & /patients & List patients \\
POST & /patients & Create patient (role dependent) \\
GET & /appointments & List appointments \\
POST & /appointments & Create appointment \\
GET & /audit & Read audit logs (ADMIN only) \\
\bottomrule
\end{tabular}

\section{Database Schema (High Level)}
\begin{itemize}[leftmargin=1.2em]
  \item \textbf{users}: user accounts and roles
  \item \textbf{patients}: patient profiles
  \item \textbf{appointments}: appointment records linking doctor and patient
  \item \textbf{audit\_logs}: security and operational event traces
\end{itemize}

\section{Screenshots}

\subsection{How to Add Screenshots}
Place your images in a folder named \texttt{screens/} next to this \texttt{.tex} file.

Example:
\begin{lstlisting}
project-doc/
  main.tex
  screens/
    login.png
    dashboard.png
    rate-limit.png
    rbac-deny.png
    audit-admin.png
\end{lstlisting}

\subsection{Screenshot 1: Login Page}
\begin{figure}[H]
  \centering
  \includegraphics[width=0.92\textwidth]{screens/login.png}
  \caption{Login page used to authenticate users and start a secure session.}
\end{figure}

\subsection{Screenshot 2: Dashboard}
\begin{figure}[H]
  \centering
  \includegraphics[width=0.92\textwidth]{screens/dashboard.png}
  \caption{Dashboard view showing key actions and navigation.}
\end{figure}

\subsection{Screenshot 3: Rate Limiting (Login Protection)}
\begin{figure}[H]
  \centering
  \includegraphics[width=0.92\textwidth]{screens/rate-limit.png}
  \caption{Rate limiting behavior: repeated login attempts from the same IP are temporarily blocked.}
\end{figure}

\subsection{Screenshot 4: RBAC Restriction}
\begin{figure}[H]
  \centering
  \includegraphics[width=0.92\textwidth]{screens/rbac-deny.png}
  \caption{Role-based access control: restricted actions are blocked server-side even if attempted from the UI.}
\end{figure}

\subsection{Screenshot 5: Audit Logs (ADMIN Only)}
\begin{figure}[H]
  \centering
  \includegraphics[width=0.92\textwidth]{screens/audit-admin.png}
  \caption{Audit logs page accessible only by ADMIN, showing security-relevant events.}
\end{figure}

\section{How to Run the Project (Short)}

\subsection{Backend}
\begin{lstlisting}
# install
npm install

# run
npm run dev
\end{lstlisting}

\subsection{Database}
\begin{itemize}[leftmargin=1.2em]
  \item Configure PostgreSQL connection in \texttt{.env}.
  \item Apply migrations / create tables as provided in the repository.
\end{itemize}

\subsection{Frontend}
\begin{lstlisting}
# If frontend is static
Open index.html or run a local server

# If React/Next
npm install
npm run dev
\end{lstlisting}

\section{Conclusion}

This project demonstrates the ability to build a full stack TypeScript application with a production mindset, including security controls (JWT, RBAC, validation/sanitization, rate limiting, audit logs) and PostgreSQL persistence.

\bigskip
\noindent\textbf{GitHub Repository:} \href{https://github.com/marina1815/bootcamp}{https://github.com/marina1815/bootcamp}

\end{document}
